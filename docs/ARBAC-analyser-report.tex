\documentclass{article}
\usepackage[utf8]{inputenc}
\usepackage{listings}
\usepackage{tabularx}
\usepackage{hyperref}
\hypersetup{
    colorlinks=true,
    urlcolor=blue,
}

\title{ARBAC analyser}
\author{Giacomo Rosin (875724)}
\date{\today}

\renewcommand{\arraystretch}{1.5}

\begin{document}

\maketitle

\sloppy         % fix overfull \hbox


\section{Introduction}
The goal of the assignment was to build an ARBAC (Administrative Role Based Access Control)
analyser for small policies, able to parse a role reachability problem specification and return
its solution (true or false).


\section{Usage}
The ARBAC analyser can be used as follows:
\begin{lstlisting}[]
 python3 arbac-analyser.py [policy.arbac]
\end{lstlisting}

\noindent
When the path to a policy is given as parameter, the analyser will read the specified file:
\begin{lstlisting}[]
 python3 arbac-analyser.py ./policies/policy1.arbac
\end{lstlisting}

\noindent
When no path is provided, the analyser will read the ARBAC specification from standard input:
\begin{lstlisting}[]
 cat ./policies/policy1.arbac | python3 arbac-analyser.py
\end{lstlisting}

\noindent
Note that this program need a version of Python greater or equal to 3.8, and the used
third party packages (only \lstinline[columns=fixed]{lark}) need to be installed.
You can obtain them by running:
\begin{lstlisting}[]
 pip install -r requirements.txt
\end{lstlisting}


\section{Implementation details}

\subsection{Parsing}
The ARBAC reachability problem specifications are parsed using a context-free grammar written
using \href{https://github.com/lark-parser/lark}{Lark} (a parsing toolkit for Python),
that uses \href{https://www.wikiwand.com/en/Extended_Backus-Naur_form}{EBNF}-inspired grammars.
If the parsing is successful, a parse tree is generated. By defining a subclass of a
\lstinline[columns=fixed]{lark.Transformer} you can transform the parse tree into a
data structure of your choice.
The grammar for the ARBAC role reachability specification language is defined in the
\lstinline[columns=fixed]{arbac_analyser/parser/arbac.lark} file, and the related
Python code is in the \lstinline[columns=fixed]{arbac_analyser/parser/arbac_parser.py} file.

The data structures used to store the parsed information are defined in the
\lstinline[columns=fixed]{arbac_analyser/types/arbac.py} file.
There is only one thing worth noting about data structures.
The user-to-role assignment is modelled through a set, in particular an immutable set,
of user-to-role associations.
A set is used, and not a list, since the ordering of the user-to-role association
is not important, and thus the semantic of the == operator is the one wanted
(same elements contained, regardless of the order). Other more efficient data structures
could be used, but set was the simpler one.
An immutable set (a Python "\lstinline[columns=fixed]{frozenset}") is used because,
later, the set of user-to-role assignments needs to be put into another set (the "visited" set),
for the role reachability algorithm (normal set can't be stored in another set). \\
The other data structures are quite straightforward to understand.

\subsection{Pruning algorithms}
Role reachability is a PSPACE-complete problem. So you need to use some technique
to deal with complex ARBAC policies.
To simplify the input policies, some pruning algorithm have been implemented.
Pruning algorithms permit to reduce the search space, without restricting the ARBAC model,
and without relying on approximate analysis techniques.
The pruning algorithms implemented are the following:
\begin{itemize}
\item \emph{Forward slicing}: Computes an over-approximation of the reachable roles,
and then simplifies the ARBAC system according to it, in a way to preserve the
solution to the role reachability problem.

\item \emph{Backward slicing}: Computes an over-approximation of the relevant roles to
assign the goal, and then simplifies the ARBAC system according to it, in a way to preserve
the solution to the role reachability problem.

\item \emph{Slicing}: Applies repetitively the forward slicing algorithm, followed by the backward
slicing algorithm, until the ARBAC system stabilises to a fixed point.
\end{itemize}

\subsection{Role reachability}
The algorithm implemented to solve role reachability searches all the possible user-to-role
assignments of the given ARBAC system, stopping when no new states are found, returning false,
or when a state that contains the goal role is found, returning true.
To visit all the state space, the implemented algorithm keeps a queue of user-to-role assignments
to analyse and a set of visited user-to-role assignments (set has been choosen since the Python
implementation can check the presence of an item in O(1) on average, while still keeping the
insertion time O(1) on average). At the beginning, the queue contains
only the initial state. At every iteration over the queue, an user-to-role assignment is extracted
and, if it was not already visited, the presence of a user with the goal role is tested. Then if it is
not present, the state is added to the list of visited states and all the possible reachable
states are generated, and added to the queue. To generate all the possible new user-to-role
assignments reachable from the current user-to-role assignment, for each pair target user
and policy rule (can assign or can revoke) a new user-to-role assignment is generated if
the preconditions to apply the rule are met.


\section{Results}
In the following table are listed the solutions to the 8 reachability problems
(\lstinline[columns=fixed]|policies/policy{1..8}.arbac|), along with
the time and the memory needed to solve them.

\begin{table}[h!]
\centering
\begin{tabularx}{1\textwidth} {
  | >{\centering\arraybackslash}X
  | >{\centering\arraybackslash}X
  | >{\centering\arraybackslash}X
  | >{\centering\arraybackslash}X | }
 \hline
 \textbf{Policy} & \textbf{Result} & \textbf{Time (s)} & \textbf{Memory (Mb)} \\
 \hline
 Policy 1 & Reachable & 0.30 & 39.044 \\
 \hline
 Policy 2 & Not reachable & 28.28 & 215.352 \\
 \hline
 Policy 3 & Reachable & 0.09 & 17.916 \\
 \hline
 Policy 4 & Reachable & 0.40 & 50.336 \\
 \hline
 Policy 5 & Not reachable & 390.06 & 1492.268 \\
 \hline
 Policy 6 & Reachable & 0.10 & 20.400 \\
 \hline
 Policy 7 & Reachable & 0.29 & 32.560 \\
 \hline
 Policy 8 & Not reachable & 379.49 & 1492.440 \\
 \hline
\end{tabularx}
\caption{Analysis of the 8 policies}
\label{tab:results}
\end{table}

\noindent
Time and memory usage measured using Gnu time (\lstinline[columns=fixed]{/usr/bin/time})
and averaged over 3 runs on an Intel i5-8400.

\end{document}
